\documentclass[12pt]{article}
\usepackage[utf8]{inputenc}
\usepackage{fancyhdr}
\usepackage{xcolor}
\usepackage{hyperref}
\usepackage{graphicx}
\usepackage{svg}
\usepackage{fontspec}
\setmainfont[Path=fonts/IBM-Plex-Sans/, BoldFont=IBMPlexSans-Bold.otf, ItalicFont=IBMPlexSans-Italic.otf, BoldItalicFont=IBMPlexSans-BoldItalic.otf]{IBMPlexSans-Regular.otf}


% Turn on the style
\pagestyle{fancy}
% Clear the header and footer
\fancyhead{}
\fancyfoot{}
\renewcommand{\headrulewidth}{0pt}
\renewcommand{\footrulewidth}{0pt}
% Set the right side of the footer to be the page number
\fancyfoot[R]{\thepage}
% style links
\hypersetup{
    colorlinks,
    citecolor=black,
    filecolor=black,
    linkcolor=black,
    urlcolor=blue
}
% general settings
\graphicspath{ {./images/} }
\setlength{\parindent}{0cm}

% custom macros
\newcommand{\betterParagraph}[1]{\paragraph{#1}\mbox{}\\}

\title{owoLED Technical Documentation}
\date{2020}
\author{молния Gang}


\begin{document}

\begin{titlepage}
\vspace*{\fill}
\begin{center}
\Huge{
\textbf{\textcolor{red}{o}}
\textbf{\textcolor{orange}{w}}
\textbf{\textcolor{yellow}{o}}
\textbf{\textcolor{green}{L}}
\textbf{\textcolor{blue}{E}}
\textbf{\textcolor{purple}{D}}
}

\Huge{Technical Documentation}

\vspace{1em}

\LARGE{молния Gang}

\vspace{0.5em}

\Large{2020}
\end{center}
\vspace*{\fill}
\end{titlepage}


\tableofcontents

\newpage

\section*{Preface}
\addcontentsline{toc}{section}{Preface}
We had a bunch of \textit{WS2813B} Led\-strips lying around and wanted to make them go blinky with an arduino.
The thing is we really don't like the Arduino Toolchain and code that includes their headers that do random magic.
While looking into alternatives we discovered that most existing solutions weren't very versatile and relied on  copying third party code aroun.
\bigbreak
This Document aims to provide all necesary Infomration that you need to use this marvelous library, that is free of Arduino bloat. 

\betterParagraph{Source Code}
The Source code of the Project and this doccument can be found at \url{https://git.chaostreffbern.ch/molniya-gang/owoled/}.

\betterParagraph{License}
This Document is Licensed under the \href{https://creativecommons.org/licenses/by-nc-sa/4.0/}{Creative Commons Attribution-NonCommercial-ShareAlike 4.0 International} License.
\bigbreak
\includegraphics{cc-by-nc-sa_icon.png}
\bigbreak
\includesvg{./images/cc-by-nc-sa_icon.svg}

\newpage
\section{Project Overview}

\newpage
\section{Using owoLED}

\newpage
\section{Examples}

\newpage
\input{build/api}


\end{document}
